\documentclass[aspectratio=169]{beamer}
\usetheme[numbering=fraction]{metropolis}
%% Layout options
\setbeamertemplate{frame footer}{\inserttitle} %% Footer with title.
\setbeamerfont{page number in head/foot}{size=\tiny}
\setbeamertemplate{bibliography item}{\insertbiblabel} %% Make [1] work in bibliography.
\setbeamertemplate{caption}{\raggedright\insertcaption\par} %% Remove 'Figure X' from caption.
\setbeamercolor{background canvas}{bg=white}

\usepackage{verbatimbox, stackengine}
\usepackage{listings}

\title{Connecting CoreDNS to Envoy's Discovery Control Plane}
\date{April/March 2020}
\author{Miek Gieben (miek@miek.nl) \and Michael Grosser (xxxk@yyy.zzz)}
\institute{Centre for protobuf Nerding}

\lstdefinelanguage{proto}{
    morekeywords={version_info, type_url, DiscoveryRequest, DiscoveryResponse},
    basicstyle=\small,
}

\begin{document}
    \let\oldfootnotesize\footnotesize
    \renewcommand*{\footnotesize}{\oldfootnotesize\tiny}

    \maketitle

    \section{What's Envoy and Envoy's Discovery Protocol?}
    \begin{frame}{Envoy and CoreDNS}
        \begin{itemize}
            \item Envoy is L7 HTTP server/proxy/load balancer written in C++
            \item Started in Lyft - it does things right
            \item Among others it's dynamic configuration (protocol) is good
        \end{itemize}

        \begin{itemize}
            \item CoreDNS is a DNS server written in Go\footnote{Originally based off of Caddy, \url{https://caddyserver.com}}
            \item Community project
            \item Using plugin architecture, making it extremely flexible
        \end{itemize}
    \end{frame}

    \begin{frame}{Traditional Load Balancing}
        \includegraphics[scale=0.45]{traditional_lb.pdf}

        Traffic traverses the proxy \emph{twice}.
    \end{frame}

    \begin{frame}{Lookaside Load Balancing}
        \begin{columns}
            \begin{column}{0.7\textwidth}
                \includegraphics[scale=0.5]{lookaside_lb.pdf}

                Client recieves assignments and \emph{directly} connects to endpoints.
            \end{column}

            \begin{column}{0.3\textwidth}
                Clients can be L7 proxy, RPC impl, DNS, etc.
            \end{column}
        \end{columns}
    \end{frame}

    \begin{frame}{Management Server: Envoy's Discovery Protocol}
        \begin{itemize}
            \item Envoy is dynamically configurable using \emph{Discovery Services} (DS)
            \item It can configure
            \begin{itemize}
                \item listeners (open new sockets)$\rightarrow$ Listeners Discovery Service (\emph{L}DS)
                \item routes (map \texttt{/api} to \texttt{X}, etc.) $\rightarrow$ Route Discovery Service (\emph{R}DS)
                \item clusters (a name + endpoints) $\rightarrow$ Cluster Discovery Service (\emph{C}DS)
                \item endpoints (address:port) $\rightarrow$ Endpoint Discovery Service (\emph{E}DS)
            \end{itemize}
        \end{itemize}

        All these DSs together are called {\bf xDS}, we care about CDS/EDS.

        Bidirectional stream variant $\rightarrow$ Aggregated Discovery Service (\emph{A}DS)
    \end{frame}

    \begin{frame}{xDS Protocol}
        Simple (~2 gRPC messages), packed with features. TODO(miek): more.
        \begin{columns}[T]
            \begin{column}{0.5\textwidth}
                \lstinputlisting[language=proto]{discovery-request.proto}
            \end{column}
            \begin{column}{0.5\textwidth}
                \lstinputlisting[language=proto]{discovery-response.proto}
            \end{column}
        \end{columns}

        You either:

        \begin{itemize}
            \item call a service, i.e. \texttt{FetchEndpoints} (xDS)
            \item bidirectional stream + indicate type \texttt{type\_url} (ADS)
        \end{itemize}
    \end{frame}

    \begin{frame}{Why Is xDS a Good Protocol}
        \begin{enumerate}
            \item It's a sane and simple protocol
            \item Emerging as industry standard?
            \item gRPC itself is moving towards it (with goodies like load reporting and health checking)
        \end{enumerate}
        If you don't have a fancy protocol (gRPC), why not use DNS for all those "legacy" protocols?
    \end{frame}

    \begin{frame}{Why do you want this?}
        We have (Kubernetes) clusters. Treat them as failure domains.
        \begin{enumerate}
            \item Services breaks in single cluster; drain service
            \item Entire cluster fails; drain \emph{all} services
        \end{enumerate}
        Just external DNS won't cut it. You need a control plane.
    \end{frame}

    \begin{frame}{Wait? Another Service Mesh?}
        No, this is not a service mesh; these are simple, single use components (KISS and Unix principle apply). That you can use
        to build something
        \begin{enumerate}
            \item You can understand
            \item Implement a global software load balancer
        \end{enumerate}
    \end{frame}


    \section{CoreDNS with \emph{traffic}}

    \begin{frame}{Caveats}
        This is DNS, you're not changing actual routing
        At mercy of clients holding to IPs
        With 5s TTL you should so sharp drop off in 5 - 10m You're doing it to protect client; if they don't care; they will get broken
    \end{frame}

    \begin{frame}{Setup for Demo}
        \begin{columns}[T]
            \begin{column}{0.2\textwidth}
                \includegraphics[scale=0.35]{lookaside_lb.pdf}
            \end{column}
            \begin{column}{0.8\textwidth}
                \includegraphics[scale=0.6]{xds-overview.pdf}
            \end{column}
        \end{columns}
    \end{frame}

    \begin{frame}{}
        \begin{description}
            \item[Health Check Server]\ \\
                 Not available in this presentation (using \texttt{xdsctl})
            \item[Management Server]\ \\
                 xds - implements Envoy xDS/ADS v3 \url{github.com/miekg/xds}
            \item[cli]\ \\
                 xdsctl - CLI to manipulate \texttt{xds}, this is what oncallers would use -
                 implements Envoy xDS v3 client side \url{github.com/miekg/xds/tree/master/cmd/xdsctl}
            \item[dns]\ \\
                 CoreDNS - with \emph{traffic} plugin -
                 implements Envoy xDS v3 ADS \url{github.com/coredns/coredns/tree/traffic}
        \end{description}
    \end{frame}

    \section{Demo}
    \begin{frame}[fragile]
        \begin{verbatim}
% xds
[INFO] Initialized cache with 'v1' of 1 cluster parsed
from directory: "."
[INFO] Management server listening on :18000
        \end{verbatim}

        \begin{verbatim}
% xdsctl ls
CLUSTER  VERSION  TYPE  METADATA
web      1        EDS
        \end{verbatim}
    \end{frame}

    \begin{frame}[fragile]
        \begin{verbatim}
% xdsctl ls web
CLUSTER  VERSION  ENDPOINT        LOCALITY  HEALTH   WEIGHT
web      1        192.168.1.1:80  us        HEALTHY  0
web      1        192.168.1.2:80  us        HEALTHY  0
web      1        172.16.0.1:80   eu        HEALTHY  0
web      1        172.16.0.2:80   eu        HEALTHY  0
        \end{verbatim}

    \end{frame}

    \begin{frame}[fragile]

        \begin{verbbox}
% xdsctl drain web 192.168.1.1:80
% xdsctl ls web
CLUSTER  VERSION  ENDPOINT        LOCALITY  HEALTH    WEIGHT
web      2        192.168.1.1:80  us        DRAINING  0
web      2        192.168.1.2:80  us        HEALTHY   0
web      2        172.16.0.1:80   eu        HEALTHY   0
web      2        172.16.0.2:80   eu        HEALTHY   0
        \end{verbbox}
        \stackinset{l}{53pt}{b}{12pt}{{\framebox(55,78){}}}{%
        \stackinset{l}{285pt}{b}{58pt}{{\framebox(62,18){}}}{\theverbbox}%
        }

    \end{frame}

    \begin{frame}[fragile]

        \begin{verbbox}
% xdsctl undrain web 192.168.1.1:80
% xdsctl ls web
CLUSTER  VERSION  ENDPOINT        LOCALITY  HEALTH   WEIGHT
web      3        192.168.1.1:80  us        UNKNOWN  0
web      3        192.168.1.2:80  us        HEALTHY  0
web      3        172.16.0.1:80   eu        HEALTHY  0
web      3        172.16.0.2:80   eu        HEALTHY  0
        \end{verbbox}
        \stackinset{l}{53pt}{b}{12pt}{{\framebox(55,78){}}}{%
        \stackinset{l}{285pt}{b}{58pt}{{\framebox(62,18){}}}{\theverbbox}%
        }

    \end{frame}

    \begin{frame}{Remaining Items}
        Work that remains consists out of
        \begin{itemize}
            \item Writing Go binary for health checking
            \item Localization support in \texttt{xds} and \emph{traffic}
            \item Fail safe: all endpoints are unhealtly $\rightarrow$ assume HC failure?
            \item Drain state, while discovering new endpoints.
            \item Authn/authz?
            \item Canarying - done as "global" boolean in DiscoveryResponse, implies whole cluster canarying? What does that mean?
        \end{itemize}

        DNS disobeyers. Clients don't refresh their assignment, so will break at some point.
        This is why TTL was invented.
        Delays and interaction with DNS TTL and our 10s cycle.
    \end{frame}

    \begin{frame}{Other options}
        \begin{itemize}
            \item Of course you don't have to load balance?
            \item Once in xDS form you can change \emph{traffic} yourself to do what you want, just serve
                    the data -> external DNS done.
        \end{itemize}
    \end{frame}

\end{document}
